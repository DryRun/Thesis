\chapter{Introduction}
% PLANNING
% 	- What is contained in this thesis?
% 		- Two searches for evidence of phenomena beyond the Standard Model in proton-proton collisions at \sqrt{s}=8 TeV.
%		- Dataset consists of 20.3 fb^-1 (=how many inelastic collisions?) of proton-proton collisions recorded by the ATLAS detector at the Large Hadron Collider.
%		- The searches utilize events containing at least three charged leptons. Such events are produced relatively rarely in Standard Model processes at the LHC, as most of the collisions proceed via the strong interaction. 
% 	- Motivation
%		- The Standard Model has described most observed phenomena in high energy particle physics experiments. 
%		- A small list of observations, largely from astrophysics, fall outside its purview(?). Particularly, neutrino mass disagrees with the Standard Model, and dark matter appears to be a particle not included in the Standard Model.
%		- These discrepancies, as well as several theoretical concerns discussed more in chapter 2, have inspired a wide effort to search for physics beyond the Standard Model. 
%	- Experimental techniques
%		- Today's particle physics experiments are broadly classified into three "frontiers" of experimental techniques. 
%		- The intensity frontier uses intense particle beams to collect high statistics. This is particularly useful for studying neutrinos, which interact very weakly with conventional matter.
%		- The cosmological frontier studies the composition of the universe and attempt to describe the evolution from the Big Bang to the current epoch. While this gives access to perhaps the largest energy range, the set of observations available to Earth makes life difficult.
%		- Finally, the energy frontier collides particles with high energy, attempting to create the new particles directly in a laboratory setting. The LHC is the flagship experiment. It is constructed to explore particle physics in great detail at the weak scale.
% 	- Trilepton searches.

% Particle physics has seen the experimental verification of the Standard Model. Verified that we exist in a world of broken symmetry, where the Higgs field condensate is responsible for the mass of the known fundamental particles. All the parameters are known, although not all of the predictions (interactions) have yet been observed. 

% A small but compelling set of inconsistencies and philosophical shortcomings have generated a number of candidates for theories that extend the Standard model. The LHC has the power to explore a subset, or a part of the parameter space, of these many models. Energy and luminosity. 

% This dissertation presents a search using rare multilepton processes with the ATLAS detector at the LHC. 


This dissertation presents two searches for phenomena beyond the Standard Model in proton-proton collision data with a center-of-mass collision energy of $\sqrt{s}=\SI{8}{\tera\electronvolt}$. The dataset contains $\SI{20.3}{\femto\barn\tothe{-1}}$ of integrated luminosity recorded by the ATLAS detector at the Large Hadron Collider (LHC) in 2012. Both searches look for an excess above the predicted Standard Model backgrounds of events containing at least three charged leptons. Such events are produced rarely in Standard Model processes, and consequently provide a useful low-background sample in which to search for new physics. 

Since its development in the 1970s, the Standard Model of particle physics has successfully described most phenomena observed in high energy physics. With the discovery of the Higgs boson at the LHC in 2012, all of the Standard Model particles have been observed, with their properties and interactions largely agreeing with predictions. A small number of observations, however, cannot be explained by the Standard Model; neutrino oscillations, dark matter, and the overabundance of matter over antimatter in the universe suggest that the content of the Standard Model is incomplete. Together with several theoretical concerns, these discrepancies have inspired a multifaceted effort to discover and understand physics beyond the Standard Model. The Standard Model, its shortcomings, and several proposed remedies are described in chapter~\ref{ch:theory}.

The experimental techniques of particle physics are roughly divided into three ``frontiers'': intensity, cosmology, and energy. The intensity frontier investigates rare processes using intense particle beams. The cosmic frontier analyzes the contents of the universe, such as the distribution of matter or the cosmic microwave background, to determine the physics responsible for the evolution of the universe from the Big Bang to today. The energy frontier uses particle accelerators to produce new particles in high-energy collisions. The three frontiers offer complementary approaches to searching for beyond the Standard Model phenomena.

The Large Hadron Collider, or LHC, is the current flagship experiment of the energy frontier, capable of producing proton-proton collisions with a center-of-mass energy of up to $\sqrt{s}=\SI{14}{\tera\electronvolt}$. The first data-taking run of the LHC, spanning 2011--2012, delivered integrated luminosities of $\int \mathcal{L}\, \mathrm{d}t=\SI{5.46}{\femto\barn\tothe{-1}}$ at $\sqrt{s}=\SI{7}{\tera\electronvolt}$ and $\int \mathcal{L}\, \mathrm{d}t=\SI{22.8}{\femto\barn\tothe{-1}}$ at $\sqrt{s}=\SI{8}{\tera\electronvolt}$ to the ATLAS detector, one of two general-purpose detectors at the LHC. The LHC and the ATLAS detector are described in chapter~\ref{ch:experiment}. The data collected are potentially sensitive to new phenomena up to roughly the TeV scale. The measurement of the integrated luminosity and the algorithms used to process the data are described in chapters~\ref{ch:luminosity} and \ref{ch:event-reconstruction}, respectively.

Events containing three or more charged leptons are a useful probe of new physics due to the small expected Standard Model backgrounds. The backgrounds and the techniques for their estimation are described in chapter~\ref{ch:backgrounds}.  This dissertation presents two signature-driven searches for new physics using trilepton events. The first, presented in chapter~\ref{ch:model-independent-trilepton-search}, is a generic search for deviations from Standard Model predictions in many signal regions sensitive to new physics. The second, presented in chapter~\ref{ch:trilepton-resonance-search}, searches for resonant trilepton production via an intermediate $Z$ boson. In both cases, no significant deviations from Standard Model predictions are observed. The data are used to set limits on several models of new physics. 
