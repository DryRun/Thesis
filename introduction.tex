\chapter{Introduction}
% PLANNING
% 	- What is contained in this thesis?
% 		- Two searches for evidence of phenomena beyond the Standard Model in proton-proton collisions at \sqrt{s}=8 TeV.
%		- Dataset consists of 20.3 fb^-1 (=how many inelastic collisions?) of proton-proton collisions recorded by the ATLAS detector at the Large Hadron Collider.
%		- The searches utilize events containing at least three charged leptons. Such events are produced relatively rarely in Standard Model processes at the LHC, as most of the collisions proceed via the strong interaction. 
% 	- Recent progress in the field
%		- Run I of the LHC was an extremely fruitful dataset for the Standard Model. 
%		- Verified previously discovered particles, particularly the W, Z, and top quark.
%		- Discovered the Higgs boson.
%	- Looking ahead
%		- With the unprecendented center-of-mass energy comes the possibility
% - Explain the analyses to be presented.

% Particle physics has seen the experimental verification of the Standard Model. Verified that we exist in a world of broken symmetry, where the Higgs field condensate is responsible for the mass of the known fundamental particles. All the parameters are known, although not all of the predictions (interactions) have yet been observed. 

% A small but compelling set of inconsistencies and philosophical shortcomings have generated a number of candidates for theories that extend the Standard model. The LHC has the power to explore a subset, or a part of the parameter space, of these many models. Energy and luminosity. 

% This dissertation presents a search using rare multilepton processes with the ATLAS detector at the LHC. 


This dissertation presents two searches for phenomena beyond the Standard Model in proton-proton collision data with a center-of-mass collision energy of $\sqrt{s}=\SI{8}{\tera\electronvolt}$. The dataset contains $\SI{20.3}{\femto\barn\tothe{-1}}$ of integrated luminosity recorded by the ATLAS detector at the Large Hadron Collider (LHC) in 2012. Both searches look for an excess above the predicted Standard Model backgrounds of events containing at least three charged leptons. Such events are produced rarely in Standard Model processes, and consequently provide a useful low-background sample in which to search for new physics. 

The first data-taking run of the LHC, spanning 2011-2012, was enormously fruitful for Standard Model physics. The data quickly confirmed the existence of previously discovered particles, including the $W$ and $Z$ bosons and the top quark, and soon revealed the existence of the long-sought Higgs boson. At the same time, the Standard Model has several known deficiencies, which have inspired a large number of theories of new physics, described in chapter~\ref{ch:theory}. Though such new physics need not manifest at the energy scales of current experiments, the LHC and the ATLAS detector, described in chapter~\ref{ch:experiment}, probe an unprecedented energy scale. The data collected in 2011 and 2012 are capable of confronting many models of new physics in qualitative interesting regions of their parameter spaces. The measurement of the integrated luminosity and the algorithms used to process the data are described in chapters~\ref{ch:luminosity} and \ref{ch:reconstruction}, respectively. 

Events containing three or more charged leptons are a useful probe of new physics due to the small expected Standard Model backgrounds. The backgrounds and the techniques for their estimation are described in chapter~\ref{ch:backgrounds}.  This dissertation presents two signature-driven searches for new physics using trilepton events. The first, presented in chapter~\ref{ch:modelindependent}, is a generic search for deviations from Standard Model predictions in many signal regions sensitive to new physics. The second, presented in chapter~\ref{resonance}, searches for resonant trilepton production via an intermediate $Z$ boson. In both cases, no significant deviations from Standard Model predictions are observed. The data are used to set limits on several models of new physics. 
