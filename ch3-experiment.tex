\chapter{The Experimental Apparatus}
\section{The Large Hadron Collider}
The Large Hadron Collider (LHC) is a particle accelerator designed to explore the physics of particles at the energy scale of electroweak symmetry breaking. The accelerator occupies a 26.7 km tunnel beneath the Switzerland-France border near Geneva, which previously housed the Large Electron Positron Collider (LEP). The geography is shown in figure~\ref{fig:LHC-map}. Protons are accelerated in two counter-rotating beams up to a design momentum of $7~\mbox{TeV}/c$. The beams collide at four interaction points (IPs), shown in figure~\ref{fig:LHC-IPs}, where four collider detectors, ATLAS, CMS, LHCb, and ALICE, analyze the remnants of the collisions.

The LHC project was approved in 1994 by the CERN Council, and construction proceeded over the ensuing 14 years. The collider detectors were constructed in parallel, beginning with the excavation of two additional caverns at IP1 and IP5 for the ATLAS and CMS detectors (LHCb and ALICE occupied the existing caverns at IP2 and IP8, which previously housed the DELPHI and L3 LEP experiments). The first beam was circulated on 10 September 2008; however, on 19 September, the LHC sustained severe damage due to an incident stemming from a faulty joint between magnets\footnote{A postmortem analysis implicated a bad splice between the superconducting cables of adjacent magnets as the source of the incident, with a resistance about $10^3$ times above specification. The joint melted, and 275~MJ of energy in the magnets dissipated in electric arcs, which vaporized beam pipes and breached the cryogenic vessel containing the magnets. A large amount of liquid helium entered the vacuum vessel and heated rapidly, breaking several vacuum barriers of the cryostats with a force of up to 56 tons. Ultimately, 30 dipoles and 7 quadrupoles were damaged beyond repair, and another 9 dipoles and 7 quadrupoles required repairs; 9 magnet interconnections were destroyed; 26 magnets were pushed down the tunnel; 276~MJ of energy were dissipated in electrical faults and arcs; 6 tons of helium were lost; and 2.8~km of both beam pipes were contaminated with fragments of insulation, with 1~km also contaminated with soot from molten copper and insulation.~\cite{Rossi:2010el}}. Repairs took an extra year, and the energy of the beams was reduced to $3.5-4~\mbox{TeV}$ for the first data-taking run, to mitigate the risk of another possible faulty joint. 

Proton-proton collisions at a center-of-mass energy of $\sqrt{s}=7~\mbox{TeV}$ commenced in early 2010. The LHC delivered an integrated luminosity of $\int L dt=48.1~\mbox{pb}^{-1}$ to the ATLAS detector in 2010, and $\int L dt=5.46~\mbox{fb}^{-1}$ in 2011. In 2012, the collision energy was increased to $\sqrt{s}=8~\mbox{TeV}$, and a dataset of $\int L dt=22.8~\mbox{fb}^{-1}$ was delivered. 

\subsection{Accelerator Parameters}
From the experiments' point of view, there are two main parameters to optimize in order to maximize sensitivity to new physics: the collision energy, $\sqrt{s}$, and the integrated luminosity, $\int L dt$. The collision energy is limited to $\sqrt{s}=14~\mbox{TeV}$ by the bending power of the dipole magnets, which have a field strength of $8.73~\mbox{T}$; however, due to the faulty splice design mentioned above, the energy was limited to $\sqrt{s}=7-8~\mbox{TeV}$ in Run I. 

The optimization of luminosity is somewhat more complicated.  The luminosity of the colliding beams is given by:
\begin{equation}\label{eqn:lumi}
	L = \frac{n_b f_r N_1 N_2 \gamma_b}{4\pi \varepsilon_n \beta_{*}},
\end{equation}
where $n_b$ is the number of colliding bunch pairs, $f_r=11245.5~\mbox{Hz}$ is the LHC revolution frequency, $N_{1,2}$ are the number of proton in the two beams, $\gamma_b$ is the relativistic gamma factor, $\varepsilon_n$ is the normalized emittance, and $\beta_{*}$ is the beta function at the collision point. TALK ABOUT THE LIMITS ON THESE NUMBERS.

In general, a higher integrated luminosity is desired; however, a high number of simultaneous collisions, $\mu=L/n_b$, can degrade the performance of the detectors. 


\subsection{The Accelerator Complex}
The LHC itself is the last stage of a 4-part acceleration chain, shown in figure~\ref{fig:LHC-injection-chain}. Protons are first produced from hydrogen gas using an electric field to strip the electrons, and then accelerated to $50~\mbox{MeV}$ in LINAC2, a linear accelerator. The protons reach a momentum of $1.4~\mbox{GeV}/c$ in the Proton Synchrotron Booster (PSB), $25~\mbox{GeV}/c$ in the Proton Synchrotron (PS), and $450~\mbox{GeV}/c$ in the Super Proton Synchrotron (SPS), before being injected into the LHC for acceleration up to the collision energy. 

\subsubsection{Accelerator Components}
The accelerator utilizes two main components to accelerate and steer the beams: radio frequency (RF) cavities, and magnets.


%\subsection{Interaction Regions}

\subsection{Run I Performance}



\section{The ATLAS Experiment}
The ATLAS detector is a large, cylindrical collider detector located at IP1 on the LHC ring (figure~\ref{fig:LHC-IPs}). The detector measures the energy and momenta of charged and colored particles produced in the collisions provided by the LHC. It consists of several subsystems occupying a (large volume of space), weighing (x tons). Closest to the interaction region, the inner detector performs momenta measurements of charged particles by tracking their movement through a solenoidal magnetic field (section~\ref{sec:inner-detector}). Past the inner detector solenoid magnet, electromagnetic and hadronic calorimeters measure the energy of electrons, photons, and hadrons. Finally, the muon spectrometer provides additional tracking and particle identification for muons in large toroidal magnetic field. 




\subsection{Magnets}
ATLAS relies on two powerful superconducting magnets to bend the trajectories of charged particles, allowing the tracking detectors to provide measurements of their momenta. 

% Solenoid

% Toroid


\subsection{Tracking}

\subsection{Calorimetry}

\subsection{Muon System}

\subsection{Data Acquisition}

\subsection{Data Preparation/Performance}


\printbibliography