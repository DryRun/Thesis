\chapter{Conclusion}
This dissertation has presented two signature-driven searches for new physics using events with three or more charged leptons in $pp$ collisions at $\sqrt{s}=\SI{8}{\tera\electronvolt}$. The dataset was collected by the ATLAS detector at the LHC, with an integrated luminosity of $L=\SI{20.3}{\femto\barn\tothe{-1}}$. The first analysis, a model-independent search in final states with three leptons, looked for excesses above background predictions in many signal regions. The search was not optimized around specific models, but rather aimed to be broadly sensitive to nonresonant trilepton production from non-Standard Model sources. The second analysis searched for the resonant production of three electrons or muons via an intermediate $Z$ boson, scanning the trilepton mass spectrum for narrowly peaked excesses. The analysis targeted the pair production of new heavy leptons, using the additional activity from the second heavy lepton to increase the sensitivity. 

In both cases, no significant excesses above Standard Model background predictions were observed, and limits were established on a variety of models. The model-independent analysis set competitive limits on a model of doubly charged scalar particles in the case of lepton flavor-violating decays, excluding masses below $\sim\SI{400}{\giga\electronvolt}$ at 95\% CL. The trilepton resonance search established limits on the pair production of heavy leptons in the context of a vector-like leptons model and a type~III neutrino seesaw model. For the vector-like leptons model, most masses in the range $\SIrange[range-phrase=-]{114}{176}{\giga\electronvolt}$ were excluded. For the type~III seesaw model, with significantly higher production cross sections, most masses in the range $\SIrange[range-phrase=-]{100}{468}{\giga\electronvolt}$ were excluded. In addition, both searches presented results in model-independent fashion along with fiducial efficiencies to aid reinterpretations of the results in the context of other models which produce trilepton final states. 

Significant gains in sensitivity can be expected in the coming years. Following a three-year-long shutdown period, the LHC has resumed $pp$ collisions with a center-of-mass collision energy of $\sqrt{s}=\SI{13}{\tera\electronvolt}$. The increase in collision energy from $\sqrt{s}=\SI{8}{\tera\electronvolt}$ to $\SI{13}{\tera\electronvolt}$ greatly enhances the sensitivity to new phenomena; for example, the pair production cross section of vector-like leptons with $m_{\lpm}=\SI{200}{\giga\electronvolt}$ more than doubles. With targeted integrated luminosities of $\SIrange[range-phrase=-]{75}{100}{\femto\barn\tothe{-1}}$ by 2018 and $\sim\SI{350}{\femto\barn\tothe{-1}}$ by 2022, the coming data will be sensitive to new phenomena well beyond the limits set in the first data-taking run.
