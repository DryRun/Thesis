\begin{abstract}
This dissertation presents two searches for phenomena beyond the Standard Model using events with three or more charged leptons. The searches are based on $\SI{20.3}{\femto\barn\tothe{-1}}$ of proton-proton collision data with a center-of-mass energy of $\sqrt{s}=\SI{8}{\tera\electronvolt}$ collected by the ATLAS detector at the CERN Large Hadron Collider in 2012.  The first is a model-independent search for excesses beyond Standard Model expectations in many signal regions. The events are required to have least three charged leptons, of which at least two are electrons or muons, and at most one is a hadronically decaying $\tau$ lepton. The selected events are categorized based on the flavor and charge of the leptons, and the signal regions are defined using several kinematic variables sensitive to beyond the Standard Model phenomena. The second search looks for new heavy leptons decaying resonantly to three electrons or muons, two of which are produced through an intermediate $Z$ boson. The resonant decay produces a narrowly-peaked excess in the trilepton mass spectrum. In both cases, no significant excess beyond Standard Model expectations is observed, and the data are used to set limits on models of new physics. The model-independent trilepton search is used to confront a model of doubly charged scalar particles decaying to $e\tau$ or $\mu\tau$, excluding masses below $\SI{400}{\giga\electronvolt}$ at 95\% confidence level. The trilepton resonance search is used to test models of vector-like leptons and the type~III neutrino seesaw mechanism. The vector-like lepton model is excluded for most of the mass range $\SIrange[range-phrase=-]{114}{176}{\giga\electronvolt}$, while the type~III seesaw model is excluded for most the mass range $\SIrange[range-phrase=-]{100}{468}{\giga\electronvolt}$. Both searches also present tools to facilitate reinterpretations in the context of other models predicting the production of three or more charged leptons.
\end{abstract}
